\documentclass[margin,line]{res}
\usepackage{textcomp}

\oddsidemargin -.5in
\evensidemargin -.5in
\textwidth=6.0in
\itemsep=0in
\parsep=0in

\newenvironment{list1}{
  \begin{list}{\ding{113}}{%
      \setlength{\itemsep}{0in}
      \setlength{\parsep}{0in} \setlength{\parskip}{0in}
      \setlength{\topsep}{0in} \setlength{\partopsep}{0in} 
      \setlength{\leftmargin}{0.17in}}}{\end{list}}
\newenvironment{list2}{
  \begin{list}{$\bullet$}{%
      \setlength{\itemsep}{0in}
      \setlength{\parsep}{0in} \setlength{\parskip}{0in}
      \setlength{\topsep}{0in} \setlength{\partopsep}{0in} 
      \setlength{\leftmargin}{0.2in}}}{\end{list}}

\usepackage{xcolor}
\usepackage[pdftex,colorlinks=true,urlcolor=blue]{hyperref}


\begin{document}

\name{Tobin T. Fricke {\small (Citizenship: USA)} \vspace*{.1in}}

\begin{resume}
\section{\sc Contact Information}
\vspace{.05in}
\begin{tabular}{@{}p{3.5in}p{3.5in}}
LIGO Livingston Observatory             & {\it mobile:}  +1 (510) 520-9697\\
PO Box 940                              & \\
Livingston, LA 70754                    & {\it e-mail:}  \href{mailto:tfricke@ligo.caltech.edu}{tfricke@ligo.caltech.edu}
\end{tabular}

%\section{\sc Research Interests}
%Laser interferometry, general relativity, functional programming, nonlinear interactions, mathematical modeling.

%\section{\sc Current Pos.}
%{\bf Louisiana State University}, graduate student (doctoral candidate)

\section{\sc Education}
{\bf Louisiana State University}\\
%{\em Department of Physics and Astronomy} 
\vspace*{-.1in}
\begin{list1}
\item[] {\sc PhD}, Physics, in progress (expected February 2011) 
\begin{list2}
\item[] Dissertation Topic:  ``Homodyne readout for laser interferometric gravitational wave detectors''
%\item[] Advisor: Prof. Gabriela Gonz\'alez
\end{list2}
\end{list1}

{\bf University of Rochester}\\
%{\em Department of Physics and Astronomy} 
\vspace*{-.1in}
\begin{list1}
\item[] {\sc MA}, Physics, May 2006
%\begin{list2}
%\item[] Advisor:  Prof. Adrian C. Melissinos
%\end{list2}
\end{list1}

{\bf University of California, Berkeley}\\
%{\em College of Engineering} 
\vspace*{-.1in}
\begin{list1}
\item[] {\sc BS}, Electrical Engineering and Computer Science,  August 2003
\begin{list2}
\item[] minor in Mathematics
\end{list2}
\end{list1}

\section{\sc Core Skills} 
Programming: Matlab, C, C++, Python.

Mathematical modeling of physical systems.

Analog and digital electronic circuit design and construction.

Low-noise optical systems.
\section{\sc Experience}
\emph{Louisiana State University} \hfill { \textcolor{gray}{August 2008 - present} }\\
Graduate research assistant (doctoral candidate).  Installed,
commissioned, and characterized Enhanced LIGO output mode cleaner (low-noise optical system) and
DC readout system.  Designed, implemented, and evaluated real-time digital signal
processing systems and analog electronics.

\emph{California Institute of Technology} \hfill {\textcolor{gray}{January 2007 - August 2008}}\\
Visitor in Physics, and staff Research Assistant.  Designed and implemented
intensity stabilization servo for 35W 10.6 \textmu{}m $\mathrm{CO}_2$
laser. Participated in commissioning of Caltech 40-meter prototype
interferometer.


\emph{University of Rochester}  \hfill { \textcolor{gray}{August 2004 - December 2006}} \\
Lab Instructor, Intro Electromagnetism Lab.  Teaching assistant,
Physics 402
(\href{http://www.pas.rochester.edu/~tobin/teaching/phy402sp06/}{probability
  theory}) and Physics 404
(\href{http://www.pas.rochester.edu/~tobin/teaching/phy404sp06/}{linear
  spaces}).  Graduate research assistant, search for gravitational
wave stochastic background at 37 kHz (first free-spectral-range of
LIGO interferometer).

%% \emph{Institute for Geophysics and Planetary Physics (UC San Diego)} \hfill {\textcolor{gray}{Summer 2004}} \\
%% Programmer/Analyst III. Developed software (in Java) to allow the use of realtime geophysical
%% data in the Kepler scientific workflow system.

\emph{CERN Summer Student Programme} \hfill {\textcolor{gray}{Summer 2003}} \\
Summer student.  Participated in test-beam operations for the hadron
calorimeter of the Compact Muon Solenoid (CMS) experiment; attended
summer lecture series in high energy physics.

\emph{Life Sciences Division, Lawrence Berkeley National Laboratory} \hfill {\textcolor{gray}{Spring 2003}} \\
Computer scientist (trainee). Developed and evaluated computer algorithms to circumvent the phase problem in
electron microscopy (in which fourier amplitudes are available from
diffraction patterns but phases are not measurable), resulting in a
paper published in Journal of Structural Biology.

\emph{Weizmann Institute of Science} (Rehovot, Israel) \hfill {\textcolor{gray}{Summer 2002}} \\
Developed improved tools for three-dimensional reconstruction of
helical structures from electron microscopy, for use in investigation
of \emph{Agrobacterium tumefaciens} and similar structures; results
published.

\emph{The Aerospace Corporation} (El Segundo, California) \hfill {\textcolor{gray}{Summer 2000}} \\
Researched noise characteristics of infrared microbolometer array and
application of array to spectroscopy. Designed and implemented new
software architecture (in C) for data flow and systems interconnection in
an unmanned aerial vehicle multi-spectral observation platform.


\emph{Geophysical Institute - REU Program} (University of Alaska) \hfill {\textcolor{gray}{Summer 1999}} \\
Implemented a system for real-time processing of seismic sensor array data
and began integration of this system with the existing conventional
seismic location system, for enhanced seismic detector sensitivity.
Performed two weeks field work, installing Passcal instruments in the
region of Denali National Park. Results presented at AGU conference.

\emph{IEEE Micromouse robotics competition} \hfill {\textcolor{gray}{1998}} \\
Designed and built a small autonomous maze-solving robot, including a hand-built microprocessor platform (68HC11),
custom circuit boards, infrared sensors and stepper motors, and embedded software written in C and assembly language.  

%% \section{\sc Coursework}

%% \textbf{Physics}: Graduate coursework in general relativity (Carroll), quantum mechanics (Shankar), classical electromagnetism (Jackson), statistical mechanics (Callen), continuum mechanics, mathematics for physics (various), and physical optics for crystallography. Undergraduate coursework in quantum mechanics (Griffiths), particle physics (Griffiths), condensed matter physics (Kittel), statistical mechanics (Kittel), metric differential geometry (do Cormo); \textbf{Engineering/Computer Science}: automatic control/regulation, signal processing, computer architecture (P\&H), microprocessor design, operating systems, computer graphics, algorithm theory (CLRS), data structures, error correcting codes, and functional programming (SICP); \textbf{Mathematics}: linear algebra, real (Rudin) and complex analysis, abstract algebra, discrete mathematics (CS 70), numerical analysis.

\section{\sc Selected Publications}

The {\sc {Ligo}} Scientific Collaboration and the Virgo Collaboration. ``\href{http://arxiv.org/abs/0910.5772}{An upper limit on the stochastic gravitational-wave background of cosmological origin}.''  \emph{Nature} \textbf{490}, 990-994 (20 August 2009). 

Ward, R. and R. Adhikari, B. Abbott, R. Abbott, D. Barron, R. Bork, \textbf{T. Fricke}, V. Frolov, J. Heefner, A. Ivanov, O. Miyakawa, K. McKenzie, B. Slagmolen, M. Smith, R. Taylor, S. Vass, S. Waldman, A. Weinstein. ``\href{http://iopscience.iop.org/0264-9381/25/11/114030/}{DC readout experiment at the Caltech 40-meter prototype interferometer}.'' \emph{Classical and Quantum Gravity}, Vol. \textbf{25}, No. 11. (2008), 114030. 

Abu-Arish, A. and D. Frenkiel-Krispin, \textbf{T. Fricke}, T. Tzfira, V. Citovsky, S. G. Wolf, and M. Elbaum. ``\href{http://www.jbc.org/content/279/24/25359.abstract}{Three-dimensional reconstruction of Agrobacterium VirE2 protein with single-stranded DNA}.'' \emph{J. Biol. Chem.}, Vol. \textbf{279}, No. 24. (11 June 2004), pp. 25359-25363.

Spence, J. C. and U. Weierstall, \textbf{T. Fricke}, R. Glaeser, and K. Downing. ``\href{http://www.ncbi.nlm.nih.gov/sites/entrez?Db=pubmed&Cmd=ShowDetailView&TermToSearch=14643223}{Three-dimensional diffractive imaging for crystalline monolayers with one-dimensional compact support},'' \emph{J. Struct Biol}, Vol. \textbf{144}, No. 1-2. (v 2003), pp. 209-218.

\textbf{Fricke, T.} and B. Lud\"ascher, I. Altintas, K.G. Lindquist, T.S. Hansen, A. Rajasekar, F.L. Vernon, J. Orcutt (2004). Poster: ``\href{http://nibot-lab.livejournal.com/34289.html}{Integration of Kepler with ROADNet: Visual Dataflow Design with Real-time Geophysical Data}.'' \emph{Eos Trans. Amer. Geophys. U.} 2004. Also presented at \href{http://ptolemy.berkeley.edu/conferences/05/index.htm}{Sixth Biennial Ptolemy Miniconference} at UC Berkeley.

Lindquist, K. G. and R.A. Hansen, \textbf{T. Fricke}. Poster: ``\href{http://nibot-lab.livejournal.com/34345.html}{Detection of Arctic Seismicity Missed by Other Regional and Worldwide Catalogs}.'' \emph{Eos Trans. Amer. Geophys. U.}, \textbf{80}, No. 46, p. F662. Also presented at the 30th Nordic Seminar on Detection Seismology, Oct. 13-15 (1999), G\"oteborg, Sweden.

%\section{\sc Small projects}
%\textit{Relevant small projects}
%
%ICFP Programming Contest - 2006, 2007, 2009
%
%Monte Carlo investigation of the Ising Model - December 2006 \\
%\href{http://web.pas.rochester.edu/~tobin/notebook/2006/12/27/ising-paper.pdf}{\texttt{http://web.pas.rochester.edu/$\sim$tobin/notebook/2006/12/27/ising-paper.pdf}}
%
%Hoshen-kopelman algorithm for cluster identification -  April 2004 \\
%\href{http://www.ocf.berkeley.edu/~fricke/projects/hoshenkopelman/hoshenkopelman.html}{\texttt{http://www.ocf.berkeley.edu/$\sim$fricke/projects/hoshenkopelman/hoshenkopelman.html}}
%
%Wavelets for image compression - October 2000 \\
%\href{http://www.ocf.berkeley.edu/~fricke/projects/wavelet/paper.pdf}{\texttt{http://www.ocf.berkeley.edu/~fricke/projects/wavelet/paper.pdf}}

\section{\sc Honors and Awards} 
LIGO Student Fellowship (2006)

4th place, Berkeley Programming Contest (2002), 5th place (2004)

IEEE Student Branch scholarship (2002)

\section{\sc blog}
\href{http://nibot-lab.livejournal.com/}{\texttt{http://nibot-lab.livejournal.com/}}

\end{resume}

\end{document}





